
% gedit:extra-issue-commands = testissue


\documentclass[12pt]{book}
\usepackage{graphicx}
\usepackage{caption}

\title{Test Document}
\author{Knuth}
\date{April 2011}

\newcommand{\testcite}[1]{\cite{#1}}
\newcommand{\testref}[1]{(\ref{#1})}
\newcommand{\testfoo}{FOO}
\newcommand{\teste}[1]{\ensuremath{\times 10^{#1}}}
\newcommand{\testissue}[1]{\textbf{#1}}

\begin{document}
\maketitle

\testissue{need to fix this}

\chapter{Main Stuff}
\section{foo}
bar pleas see \ref{fig:fixme}

\section{bar}
baz.

Lets enjoy \teste{12}. Lets complete complete. Lets \testcite{dijkstra68}

%FIXME: foo

\section{baz}
I like \cite{dijkstra76}

\begin{figure}
    \centering
	\includegraphics[scale=1.00]{FIXME}
	\caption{Indeed}
	\label{fig:fixme}
\end{figure}

\section{cake}\label{sec:foo}
\subsection{chips}\label{sec:bar}
\subsubsection{candy}\label{sec:baz}

lets use a \testcite{dijkstra68}

see \ref{sec:foo}.

also see \testref{sec:bar}

\chapter{Another Chapter}
\section{testing}
This is a test
\subsection{another}
I can \cite{dijkstra76}

see \cite{dijkstra68}

\section{More test}


% gedit:master-filename = article.tex

\chapter{A Third Chapter}
\section{Imma Let You Finish}
This is the best \LaTeX plugin of all time.


\bibliography{article}
\bibliographystyle{IEEEtran}

\end{document}
